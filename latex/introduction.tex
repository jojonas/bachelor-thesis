\chapter{Introduction}

\section{Units and Notation}
Throughout this work, a subset of the natural unit system will be used, as it is common in high energy particle physics. The speed of light and the reduced Planck constant are fixed to $c = 1$ and $\hbar = 1$, and as such they are omitted in equations. Additionally, energy is expressed in \eV, where \unit{1}{\eV} is the energy gain of an electron which is accelerated across a \unit{1}{\volt} potential ($\unit{1}{\eV} \approx \unit{1.602 \e{-19}}{\joule}$). These conventions induce a change in units for the other dimensions, most importantly mass and momentum, both of which are notated in \GeV.

\section{The Standard Model}
The standard model represents our current understanding of elementary particles, their forces and interactions. 
Particles in the standard model can be classified into two separate classes: Fermions, which make up matter and possess a spin of \nicefrac{1}{2}, and gauge bosons, which are mediators of forces and possess an integer spin.
Three elementary forces are described in the standard model: \emph{electrodynamic} (Quantum Electrodynamics, QED), \emph{strong} (Quantum Chromo Dynamics, QCD) and \emph{weak} (Quantum Flavor Dynamics). The forces are induced by corresponding charge-like properties: electrodynamic charge, color charge and weak isospin. These charges can now be used to further subdivide fermions into quarks and leptons which will be described in the following sections.

\subsection{Leptons}
There are three charged leptons: the electron (\Pe), the muon (\Pmu) and the tau (\Ptau). They carry the electric charge of \unit{-1}{e} and participate in electrodynamic and weak interactions. For each charged lepton, there is one neutrino counterpart (\Pnue, \Pnum, \Pnut). Neutrinos do not carry electrical charge and only interact weakly. Additionally, they have no or very low mass and remain undetected in current collider experiments.
The leptons do not carry color charge and are excluded from the strong interaction.
An overview about the leptons and their masses can be found in table \ref{tbl:sm_leptons}.

\begin{table}[htbp]
	\center
	\begin{tabular}{ r | l | l | l | }
		\cline{2-4}
		& \multicolumn{3}{c|}{Leptons} \\ \cline{2-4}
		& electron (\Pe) & muon (\Pmu) & tau (\Ptau) \\ \cline{2-4}
		mass & \unit{511.0}{\keV} & \unit{105.7}{\MeV} & \unit{1.777}{\GeV} \\ \cline{2-4}
		charge & $-1$ & $-1$ & $-1$ \\ \cline{2-4}
		& \Pe neutrino (\Pnue) & \Pmu neutrino (\Pnum) & \Ptau neutrino (\Pnut) \\ \cline{2-4}
		mass & $< \unit{2}{\eV}$ & $< \unit{0.19}{\MeV}$ & $< \unit{18.2}{\MeV}$ \\ \cline{2-4}
		charge & 0 & 0 & 0 \\ \cline{2-4}
	\end{tabular}
	\caption{Leptons in the standard model\cite{PDG2014}.}
	\label{tbl:sm_leptons}
\end{table}

\subsection{Quarks}
Similarly to the leptons, quarks can be divided into three generations. The "up-type" quarks (up, down), "charm-type" quarks (charm, strange), and "top-type" quarks (top, bottom). Quarks carry a fractional electric charge of either \unit{\nicefrac{2}{3}}{e} or \unit{\nicefrac{-1}{3}}{e}.
They also carry color charges and take part in the strong interaction as well as in the weak interaction.
The three quark generations and their properties are shown in table \ref{tbl:sm_quarks}.

\begin{table}[htbp]
	\center
	\begin{tabular}{ r | l | l | l | }
		\cline{2-4}
		& \multicolumn{3}{c|}{Quarks} \\ \cline{2-4} 
		& up (\Pup) & charm (\Pcharm) & top (\Ptop) \\ \cline{2-4}
		mass & \unit{2.3}{\MeV} & \unit{1.28}{\GeV} & \unit{173.2}{\GeV} \\ \cline{2-4}
		charge & \nicefrac{2}{3} & \nicefrac{2}{3} & \nicefrac{2}{3} \\ \cline{2-4}
		& down (\Pdown) & strange (\Pstrange) & bottom (\Pbottom) \\ \cline{2-4}
		mass & \unit{4.8}{\MeV} & \unit{95}{\MeV} & \unit{4}{\GeV} \\ \cline{2-4}
		charge & \nicefrac{-1}{3} & \nicefrac{-1}{3} & \nicefrac{-1}{3} \\ \cline{2-4}
	\end{tabular}
	\caption{Quarks in the standard model\cite{PDG2014}.}
	\label{tbl:sm_quarks}
\end{table}

\subsection{Bosons}
Gauge bosons are mediators of the elementary forces. QED is mediated by the massless photon (\Pphoton), QCD by the massless gluon (\Pgluon) and the weak interaction by the massive \PZ and \PWpm bosons. They couple to the corresponding charges.
All bosons and their properties are listed in table \ref{tbl:sm_bosons}.

\begin{table}[htbp]
	\center
	\begin{tabular}{ r | l | l | l | l | }
		\cline{2-5}
		& \multicolumn{4}{c|}{Bosons} \\ \cline{2-5} 
		& Photon (\Pgamma) & Gluon (\Pgluon) & Z-Boson (\PZ) & W-Bosons (\PWpm) \\ \cline{2-5}
		mass & 0 & 0 & \unit{91.2}{\GeV} & \unit{80.4}{\GeV} \\ \cline{2-5}
		charge & 0 & 0 & 0 & $\pm 1$ \\ \cline{2-5}
		interaction & QED & QCD & weak & weak \\ \cline{2-5}
	\end{tabular}
	\caption{Bosons in the standard model\cite{PDG2014}.}
	\label{tbl:sm_bosons}
\end{table}

\subsection{Antiparticles}
For each mentioned particle, there exists one counterpart with the same mass but an opposite sign for all charge-like properties, called \emph{antiparticle}.

\section{LHC}

\section{CMS}

