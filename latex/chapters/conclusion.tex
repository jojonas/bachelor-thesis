% !TeX spellcheck = en_US
% !TeX encoding = UTF-8
% !TeX root = ../document.tex

\chapter{Conclusion}
The aim of this work was to develop a fast search algorithm for the MUSiC framework. After assessing the current performance situation, the concept of a region preselection step was considered. Multiple aspects of the algorithm were adapted to problems arising in the current implementation. The emerging solution is called \emph{Quickscan}. The Quickscan algorithm uses a less computation intensive estimator to assemble a list of candidate regions. Problems due to low statistics in the high-energy-tail of the kinematic distributions are suppressed by dedicated handling of nested regions. The original \p~value is subsequently only computed for the few collected candidates. The algorithm has been implemented and tested in this work. A value for the number of candidate regions, $\paramregions = \num{200}$ has been proposed here. It has been obtained by an optimization run over a wide parameter range. The choice was additionally validated over a separate subset of data. During this validation run, a performance gain of \SI{920}{\percent} was observed. This is mostly due to the smaller number of integrals which remain to be computed after the Quickscan selection.

Additionally, the Quickscan has not only been developed and validated, but also implemented and integrated into the existing MUSiC framework.

\enlargethispage{0.5cm}
\vspace{-0.2cm}
\section{Outlook}
Even though the result of this work increases scanning efficiency by almost \num{10} times, there is room for improvement:

The Quickscan algorithm could greatly benefit from an estimator that improves handling of regions with $\Nmc < \num{3}$. A possible solution could be to use a Poisson distribution around the number of MC events and calculate its \p~value back to a Gaussian statistic before combining with the systematical uncertainty. This could lead to a better mathematical understanding of the algorithm, especially since the ad-hoc nested region handling may turn out to be superfluous.

Furthermore, the parallelization of the general scanning algorithm could be improved. The correlated dicing of the pseudo-experiments requires communication between the worker tasks. The correlation could instead be handled by a fixed seed of the worker's pseudo-random number generators. This would make the scanning step trivially parallelizable (either over the classes or over the pseudo-experiments), enabling it to run on a computing grid instead of the 64-core MUSiC host.

