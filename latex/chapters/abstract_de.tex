% !TeX spellcheck = de_DE
% !TeX encoding = UTF-8
% !TeX root = ../document.tex

\vspace*{5mm}
\vspace*{\fill}
{\usekomafont{chapter}Kurzdarstellung}
\chapterheadendvskip

Das CMS Experiment am LHC produziert sehr große Datenmengen: pro Sekunde werden rund \SI{20}{\tera\byte} an Daten von der Detektorhardware generiert. Selbstverständlich erfordert die Auswertung viel Rechenleistung. Mithilfe mehrerer Algorithmen wird dem Detektorsignal eine physikalische Bedeutung zugewiesen, aufgrund derer modernste Analysen durchgeführt werden.

Die modellunspezifische Suche MUSiC (engl: Model Unspecific Search in CMS) ist eine Analyse, die auf einem breiten Spektrum von Endzuständen arbeitet. Dabei werden kinematische Verteilungen der Endzustände erstellt und mit Erwartungen aus Monte-Carlo-Simulationen verglichen. Die Suche nach Abweichungen macht MUSiC sensitiv gegenüber Anzeichen von Physik jenseits des Standardmodelles.

Diese Arbeit erklärt, implementiert und validiert einen zusätzlichen Arbeitsschritt der MUSiC-Analyse, der die Rechenzeit drastisch reduziert.