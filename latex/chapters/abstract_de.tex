% !TeX spellcheck = de_DE
% !TeX encoding = UTF-8
% !TeX root = ../document.tex

\vspace*{\fill}
{\usekomafont{chapter}Kurzdarstellung}
\chapterheadendvskip

Die modellunspezifische Suche MUSiC (engl.: Model Unspecific Search in CMS) wertet großen Datenmengen des CMS Experiments aus. Dabei werden in den kinematischen Distributionen Abweichungen zwischen gemessenen Zählraten und Monte-Carlo-Simulationen des Standardmodelles identifiziert. Anders als konventionelle dedizierte Analysen beschäftigt sich MUSiC nicht nur mit einigen wenigen Zerfallskanälen, sondern behandelt eine große Bandbreite an Endzuständen. 
Die Berechnung der signifikantesten Abweichung in jeder kinematischen Distribution ist verhältnismäßig rechenintensiv. Daher wird dem Algorithmus in dieser Arbeit ein zusätzlicher Auswahlschritt hinzugefügt, der die Performance deutlich erhöht und dabei das physikalische Ergebnis unbeeinflusst lässt. 
Der \emph{Quickscan} genannte Algorithmus wird in dieser Arbeit motiviert, beschrieben, optimiert und validiert. Zum Schluss beträgt der Performancegewinn ungefähr \si{920}{\%} und ermöglicht es der MUSiC-Gruppe, zukünftige Analysen schneller auszuführen.