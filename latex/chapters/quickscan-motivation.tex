% !TeX spellcheck = en_US
% !TeX encoding = UTF-8
% !TeX root = ../document.tex

\chapter{Motivation for a Fast Search Algorithm}
\label{ch:quickkscan_motivation}

The calculation of the \p~value includes integration over a series. The \p~value is calculated for each connected bin region, resulting in a runtime of $\order{n^2}$ with the number of bins in a distribution. Additionally, the runtime increases linearly with the number of classes, distributions and pseudo-experiments.

The typical number of integrals computed during a full scan of 2012 data, one kinematic variable and \num{e5} pseudo-experiments is $\order{\numrange[range-phrase=-]{e9}{e10}}$. The computation of each single integral takes about \SI{200}{\micro\second} (see table \ref{tbl:motivation_timing} in the appendix), which results in a total computation time of about \numrange{2}{20} CPU-days. The scan is automatically parallelized between worker processes and usually performed on a \num{64} core machine, thus taking \numrange{1}{9} hours.

The calculation of the integral is performed using simple adaptive integration (QAG, from the GNU Scientific Library). Since this implementation is already highly optimized, further improvement of the integral calculation is difficult.

Since the ultimate quantity of interest is the \ptilde~value, the goal  of this works fast search algorithm, called \emph{Quickscan}, is to reduce the time spent on scanning pseudo-experiments. This will be achieved by reducing the amount of candidate regions, for which the \p~value integral is calculated. 

The new algorithm shall not reduce the amount of regions too far, such that the "true" region of interest is not included anymore, since this would highly impact statistical accuracy. Thus, the deviation of \ptilde with and without Quickscan will be observed in the remainder of this work.

Shortening the total scanning time will eventually enable the MUSiC project to increase the number of pseudo experiments and remove other requirements that have been made for optimization, but have a higher impact on statistical accuracy.
