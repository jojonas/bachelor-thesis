% !TeX spellcheck = en_US
% !TeX encoding = UTF-8
% !TeX root = ../document.tex

\chapter{Motivation for a Fast Search Algorithm}

The calculation of the \p~value includes integration over a series. The \p~value is calculated for each connected bin region, resulting in a runtime of $\order{n^2}$ with the number of bins in a distribution. Additionally, the runtime increases linearly with the number of classes, distributions and pseudo-experiments.

The typical number of integrals computed during a full scan of 2012 data, one kinematic variable and $10^5$ pseudo-experiments is $\order{10^{9} - 10^{10}}$. Table \ref{tbl:motivation_timing} shows that the computation of each single integral takes about \unit{200}{\micro\second}, which results in a total computation time of about 2 to 20 CPU-days. The scan is usually performed on a 64 core machine and thus takes between 1 and 9 hours wall-time.

The calculation of the integral is performed using simple adaptive integration (QAG, from the GNU Scientific Library). Since this implementation is already highly optimized, further improvement is difficult.

The fast search algorithm, called \emph{Quickscan} in this work, aims to reduce the number of integrals by cutting down the number of region of interest candidates for which the \p~value is calculated. It shall not reduce the amount of regions so far that the "true" region of interest is not included anymore, since this would highly impact physics accuracy and will be closely examined in the remainder of this work.

Shortening the scanning time will eventually enable the MUSiC project to increase the number of pseudo experiments and remove other requirements that have been made for optimization, but have a higher impact on physics accuracy.

\begin{table}[htbp]
	\centering
	\begin{tabular}{| r | r || r |}
		\hline
		Number of integrals & Total runtime & Time per integral \\
		\hline \hline
		23110209 & \unit{5179.87}{\second} & \unit{224}{\micro\second} \\
		81734778 & \unit{16983.60}{\second} & \unit{207}{\micro\second} \\
		\hline
	\end{tabular}
	\caption{Benchmarking results of the original scanning process. The timing has been measured on a data subset, as wall-time, using a single CPU and includes setting up the process and writing out the results. The agreement of the time per integral between the runs motivates the rough estimate of \unit{200}{\micro\second} per integral.}
	\label{tbl:motivation_timing}
\end{table}