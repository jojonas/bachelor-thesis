% !TeX spellcheck = en_US
% !TeX encoding = UTF-8
% !TeX root = ../document.tex

%\chapter*{Abstract}
{\usekomafont{chapter}Abstract}
\chapterheadendvskip

The CMS experiment at the LHC produces a vast amount of data: Each second about \SI{20}{\tera\byte} of information is generated by the detector hardware. Accordingly, analysis of the data also requires a lot of computing power. Using a chain of several algorithms, the detector signal is interpreted as physical meaningful data, on which state-of-the-art analyses are performed. 

The Model Unspecific Search in CMS (MUSiC) is an analysis carried out on a wide spectrum of final states. Kinematic distributions of these final states are aggregated and compared to the expectation from Standard Model Monte Carlo simulations. By searching for deviations, MUSiC is sensitive to indications of physics beyond the standard model.

This thesis proposes, implements and validates an additional step in the MUSiC analysis, which drastically reduces the runtime.
