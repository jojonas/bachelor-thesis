% !TeX spellcheck = de_DE
% !TeX encoding = UTF-8
% !TeX root = ../document.tex

\chapter*{Danksagung}
Ich möchte mich ganz herzlich bei allen Menschen bedanken, die mir beim Erstellen dieser Arbeit zur Seite standen. 

Das behandelte Thema schlägt eine Brücke zwischen meinem Studium, der Teilchenphysik, und meinem Hobby, der Softwareentwicklung. Dafür und für die Unterstützung während des Verfassens der Arbeit danke ich ganz besonders Herrn Professor Thomas Hebbeker.

Am III. Physikalischen Institut wurde ich als Bachelor sehr herzlich aufgenommen und danke dafür allen Mitgliedern der Aachen-IIIA-CMS-Gruppe. Eine besonders gute Atmosphäre herrschte dabei innerhalb der MUSiC-Gruppe, bestehend aus Deborah Duchardt, Simon Knutzen, Tobias Pook und Andreas Albert. Durch das Beantworten hunderter Fragen hat die gesamte Physikgruppe mein Physik- und Wissenschaftsverständnis geprägt.

Explizit möchte ich Simon Knutzen und Deborah Duchardt dafür danken, dass sie mich bei der Entwicklung des Quickscans in die richtige Richtung geleitet haben und meine Arbeit im Anschluss probegelesen haben.

Des Weiteren möchte ich mich für das finale Korrekturlesen der Arbeit und für hilfreiche Hinweise und Unterstützung während der Zwischenpräsentationen bei Dr. Arnd Meyer bedanken. Ebenfalls danke ich Herrn Professor Martin Erdmann, der sich bereit erklärt hat, diese Arbeit als Zweitkorrektor zu betreuen und mich zuvor mit den Vorlesungen der Teilchenphysik für eine Bachelorarbeit in diesem Feld begeistern konnte.

Abschließend bedanke ich mich bei meiner Familie, meinen Freunden und meinen Mitbewohnern für die mentale Unterstützung, die ich in sowohl in den stressigeren letzten Monaten als auch im Rest des Studiums genießen konnte.

%Thomas Hebbeker
%Martin Erdmann
%Arnd Meyer
%MUSiC Group:
%	Simon Knutzen
%	Deborah Duchardt
%	Tobias Pook
%	Andreas Albert
%CMS Group in Aachen
%Friends \& Family